\documentclass{article}
\usepackage[utf8]{inputenc}
\usepackage{listings}
\usepackage{color}
\usepackage{mathtools}% http://ctan.org/pkg/mathtools
\definecolor{dkgreen}{rgb}{0,0.6,0}
\definecolor{gray}{rgb}{0.5,0.5,0.5}
\definecolor{mauve}{rgb}{0.58,0,0.82}

\lstset{frame=tb,
  language=Java,
  aboveskip=3mm,
  belowskip=3mm,
  showstringspaces=false,
  columns=flexible,
  basicstyle={\small\ttfamily},
  numbers=none,
  numberstyle=\tiny\color{gray},
  keywordstyle=\color{blue},
  commentstyle=\color{dkgreen},
  stringstyle=\color{mauve},
  breaklines=true,
  breakatwhitespace=true,
  tabsize=3
}

\title{Summation: thoughts and considerations from the perspective of a Computer Science undergraduate}
\author{Felipe  Augusto Morais Silva}
\date{March 2022}

\begin{document}

\maketitle
For a computer science student, the comprehension of the concepts brought by the summation notation can be described as a twist of already known concepts.
For instance, an easy and simple sum:
\begin{align*}
  A &= \sum_{i=1}^{10} i
\end{align*}
can be effortlessly converted to a code snippet like this:
\begin{lstlisting}
for(int i = 1; i <= 10; i++){
    //do something
}
\end{lstlisting}
In the example above, we can see that the sum notation is, in essence, another way of describing a concept we have already grasped. That being said, i would argue that it is almost impossible to not relate a summation to a for loop for those with a programming background.

Throughout the class we got familiar with a variety of concepts regarding sum manipulation. That being said, the major properties we should remark are:
\\%equivalent to \newline
\\
1. Constant values can freely shift between the inside and the outside of the summation
\begin{align*}
    \sum_{i=1}^{n}ki = k\sum_{i=1}^{n}i 
\end{align*}
\\
\\
2. A summation can be broken into smaller summations of the same order:
\begin{align*}
    \sum_{i=1}^{n}(a+b) &= \sum_{i=1}^{n}a + \sum_{i=1}^{n}b
\end{align*}
\begin{align*}
\sum_{i=1}^{n}(a-b) &= \sum_{i=1}^{n}a - \sum_{i=1}^{n}b
\end{align*}
\\
\\
3. The order of operations does not matter
\begin{align*}
    \sum_{i=1}^{n}a + \sum_{i=1}^{n}b - \sum_{i=1}^{n}c = \sum_{i=1}^{n}c - \sum_{i=1}^{n}a + \sum_{i=1}^{n}b
\end{align*}
\end{document}
\\ 
\\

In addition to these properties, it is also important to note that we had contact with the idea of the\href
{https://math.stackexchange.com/questions/2550043/how-to-convert-a-summation-into-a-closed-form}
{closed form} of a summation.

In essence, it is possible to observe and conclude that mathematical concepts have a huge impact and 
implication when it comes to Computer Science and its branches, thus permitting an easier abstraction of 
computational concepts, and vice-versa.
